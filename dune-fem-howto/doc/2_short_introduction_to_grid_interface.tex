This chapter gives a really short introduction to \Grid. For a more detailed explanation please read the \texttt{DUNE GRID HOWTO}
(\url{http://www.dune-project.org/doc/grid-howto/grid-howto.pdf}).

Understanding code can be sometimes difficult. The \Dune\ Coding Style document can help you to understand \Dune\ code, see 
\url{http://www.dune-project.org/doc/devel/codingstyle.html}.

The source code to all of the following examples is shipped with this documentation, see directory \lstinline!dune-femhowto/src_grid!.


\section{The \texttt{dune hello world} programm}
  \begin{lst}[File ../src\_grid/dune\_hello\_world.cc] \label{lsthelloworld} \mbox{}
    \lstinputlisting[numbers=left,numberstyle=\tiny,numbersep=5pt]{../src_grid/dune_hello_world.cc}
  \end{lst}

This little programm has a sequentiell and a parallel part. The parallel part is regarded in section \ref{parallel}. 
In fact, this program just do nothing. But you can check whether your \Dune\ installation is working. Output:
\begin{lstlisting}
Hello World! This is DUNE.
This is a sequential program.
\end{lstlisting}




\section{Getting started with your first grid}
Now we want to create our first grid:

  \begin{lst}[File ../src\_grid/gettingstarted.cc] \label{lstgettingstarted} \mbox{}
    \lstinputlisting[numbers=left,numberstyle=\tiny,numbersep=5pt]{../src_grid/gettingstarted.cc}
  \end{lst}

Here we create an SGRID (lines 17-21) for several space dimensions (lines 50,53 and 56) und print some information about these grids. 
Notice we use generic programming:
\begin{lstlisting}
template <int dim> 
struct Info 
   { ... };
\end{lstlisting}

Interesting is the number of codimensions in the several space dimensions. We just created an unitcube for the 
dimension 1, 2 and 3. 
\begin{itemize}
 \item 1D: We have one codim 0 element (the unit intervall) and two codim 1 elements (the end points of the intervall)
    \begin{lstlisting}
      => SGrid(dim=1,dimworld=1)
      level 0 codim[0]=1 codim[1]=2
      leaf    codim[0]=1 codim[1]=2
    \end{lstlisting}
 \item 2D: We have one codim 0 element (the unit square), four codim 1 elements (edges) and four codim 2 elements (vertices).
    \begin{lstlisting}
      => SGrid(dim=2,dimworld=2)
      level 0 codim[0]=1 codim[1]=4 codim[2]=4
      leaf    codim[0]=1 codim[1]=4 codim[2]=4
    \end{lstlisting}
 \item 3D: We have one codim 0 element (the unit cube), six codim 1 elements (faces), twelve codim 2 elements (edges) 
           and eight codim 3 elements (vertices).
    \begin{lstlisting}
      => SGrid(dim=3,dimworld=3)
      level 0 codim[0]=1 codim[1]=6 codim[2]=12 codim[3]=8
      leaf    codim[0]=1 codim[1]=6 codim[2]=12 codim[3]=8
    \end{lstlisting}
\end{itemize}
Because we did't refined the grid, there only exists a level 0 grid, and this level grid is the same as the leaf grid.
You have access to all these different codim elements via iterators, even after refining you have access to all elements of the different levels.

For a more detailed explanation please read the \Dune\ GRID HOWTO 
(\url{http://www.dune-project.org/doc/grid-howto/grid-howto.pdf}) 



\section{The DGF-Parser: Using different grids / macrogrids}
\Dune\ is designed to easily handle with different grid implementations. Therefore it is really easy to change the used grid.
To show this fact we take a look to the following code:

  \begin{lst}[File ../src\_grid/dgfparser.cc] \mbox{}
    \lstinputlisting[numbers=left,numberstyle=\tiny,numbersep=5pt]{../src_grid/dgfparser.cc}
  \end{lst}

By default, programs are compiled using the options \lstinline!GRIDTYPE=YASPGRID GRIDDIM=3!. If you want to change the grid type or the number of space dimensions, you just have to recompile the program again using some other options for make, for example: \lstinline!make clean GRIDTYPE=ALUGRID_SIMPLEX GRIDDIM=3!. How you have to implement this feature is shown in \lstinline!dgfGridType()!, lines 45-56.
Attention: For changing your grid in the described way, you must define \lstinline!GRIDTYPE! somewhere in an \lstinline!makefile.am! or in your configuration options during installation!

Another possibility is shown in \lstinline!dgfTest()!, lines 22-30. Here the used grid is hardcoded in the source code (line 29). This is similar to the version in Listing \ref{lstgettingstarted}. The only difference is that you use there an SGrid without macrogrids and without the DGF-Parser.

For more information: See DGF-Parser documentation (Link is below). We also refer to the list of the available grid implementations (\ref{availablegrids}).

Are you bored about these stupid unit cubes? You can change your macrogrid just by editing the corresponding \texttt{dgf} files. How this can be done is explained in the documentation of the DGF-Parser. This documentation you find in your local \Grid\ documentation, under the point \texttt{I/O - The Dune Grid Format (DGF)}, or online under \url{http://www.dune-project.org/doc/doxygen/dune-grid-html/group__DuneGridFormatParser.html}. 

You can find the example \texttt{dgf}-files from that documentation in your local \Fem\ installation, in the subdirectory \texttt{macrogrids/DGFMacrogrids}.

The program above also contains an output routine, which writes the grid information into a \texttt{vtk} file (lines 37-38 and 69-70). You can visualize your grid by opening these files with \texttt{paraview} (\url{http://www.paraview.org/}) or another VTK-Viewer.

\begin{exc}
Play with different dimensions, different grids implementations and different macrogrids, and visualize the results with \texttt{paraview}!
\end{exc}


\section{The parallel \texttt{dune\_hello\_world} programm}\label{parallel}
Now we want to run the \texttt{dune\_hello\_world} (listing \ref{lsthelloworld}) programm parallel on several computers.
\bigskip

To do that, you have to recompile the package with a modified \texttt{config.opts}. Replace in listing \ref{config.opts} \lstinline!--disable-parallel! with \lstinline!--enable-parallel! (line 20) and \lstinline!CXX=g++! with \lstinline!CXX=mpiCC! (line 27). After that, recompile the \Fem\ HOWTO with your modified \texttt{config.opts}, which ist now called \texttt{config\_parallel.opts}:
\begin{lstlisting}
  ./dune-common/bin/dunecontrol --opts=config_parallel.opts --only=dune-femhowto all
\end{lstlisting}
Notice: By using the \texttt{--only=XXXX} option, dunecontrol will only work on the module XXXX. 


\begin{enumerate}

\item Running the program on one computer:
\begin{lstlisting}
./dune_hello_world
\end{lstlisting}

Output:
\begin{lstlisting}
Hello World! This is DUNE.
I am rank 0 of 1 processes!
\end{lstlisting}
Compare that with the output of the serial version!

\item Running on several computers using the ''Simple Linux Utility for Resource Managment (SLURM)'':
\begin{lstlisting}
srun -A -n 4
mpirun nice ./dune_hello_world
exit
\end{lstlisting}
Here we use 4 machines (\lstinline!-n 4!) for running the program, resulting in an output similar to this:
\begin{lstlisting}
Hello World! This is DUNE.
Hello World! This is DUNE.
I am rank 0 of 5 processes!
Hello World! This is DUNE.
I am rank 3 of 5 processes!
I am rank 1 of 5 processes!
Hello World! This is DUNE.
I am rank 2 of 5 processes!
Hello World! This is DUNE.
I am rank 4 of 5 processes!
\end{lstlisting}
Don't forget the final \lstinline!exit! to terminate your job!!!!

\item Running on several computer without using SLURM:
\begin{lstlisting}
nice mpirun -np 4 ./dune_hello_world
\end{lstlisting}
The output will be the same as above. Using SLURM has some advantages, for example it can avoid conflicts when running several jobs on one machine. So better use it! More information: SLURM manpages (type: \lstinline!man slurm!).
\end{enumerate}

You should also use \lstinline!nice! like in the examples above to adjust scheduling priority.

Some maybe helpful commands:
\begin{itemize}
 \item \lstinline!squeue!: Returns the list of jobs managed by SLURM (and their job IDs).
 \item \lstinline!scancel <JOB ID>!: Cancels the corresponding job.
 \item \lstinline!sinfo!: Print some general information.
\end{itemize}
\bigskip

Attention: Only YaspGrid and AluGrid are able to perform parallel computations!

For real applications please read section 8 in the \Dune\ GRID HOWTO   (\url{http://www.dune-project.org/doc/grid-howto/grid-howto.pdf}).

More information: \texttt{dune-common} documentation - \texttt{Parallel Communication}  (\url{http://www.dune-project.org/doc/doxygen/dune-common-html/group__ParallelCommunication.html}).


\section{Implementation of a Finite Volume Scheme with \Grid}
You can find different implementations of a Finite Volume Scheme in the \Dune\ GRID HOWTO 
(\url{http://www.dune-project.org/doc/grid-howto/grid-howto.pdf}).

\begin{itemize}
 \item Basic scheme: Section 6.3
 \item Adaptive scheme: Section 7.2
 \item Parallel scheme: Section 8.3
\end{itemize}








